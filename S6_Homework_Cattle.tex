% Options for packages loaded elsewhere
\PassOptionsToPackage{unicode}{hyperref}
\PassOptionsToPackage{hyphens}{url}
%
\documentclass[
]{article}
\usepackage{amsmath,amssymb}
\usepackage{iftex}
\ifPDFTeX
  \usepackage[T1]{fontenc}
  \usepackage[utf8]{inputenc}
  \usepackage{textcomp} % provide euro and other symbols
\else % if luatex or xetex
  \usepackage{unicode-math} % this also loads fontspec
  \defaultfontfeatures{Scale=MatchLowercase}
  \defaultfontfeatures[\rmfamily]{Ligatures=TeX,Scale=1}
\fi
\usepackage{lmodern}
\ifPDFTeX\else
  % xetex/luatex font selection
\fi
% Use upquote if available, for straight quotes in verbatim environments
\IfFileExists{upquote.sty}{\usepackage{upquote}}{}
\IfFileExists{microtype.sty}{% use microtype if available
  \usepackage[]{microtype}
  \UseMicrotypeSet[protrusion]{basicmath} % disable protrusion for tt fonts
}{}
\makeatletter
\@ifundefined{KOMAClassName}{% if non-KOMA class
  \IfFileExists{parskip.sty}{%
    \usepackage{parskip}
  }{% else
    \setlength{\parindent}{0pt}
    \setlength{\parskip}{6pt plus 2pt minus 1pt}}
}{% if KOMA class
  \KOMAoptions{parskip=half}}
\makeatother
\usepackage{xcolor}
\usepackage[margin=1in]{geometry}
\usepackage{color}
\usepackage{fancyvrb}
\newcommand{\VerbBar}{|}
\newcommand{\VERB}{\Verb[commandchars=\\\{\}]}
\DefineVerbatimEnvironment{Highlighting}{Verbatim}{commandchars=\\\{\}}
% Add ',fontsize=\small' for more characters per line
\usepackage{framed}
\definecolor{shadecolor}{RGB}{248,248,248}
\newenvironment{Shaded}{\begin{snugshade}}{\end{snugshade}}
\newcommand{\AlertTok}[1]{\textcolor[rgb]{0.94,0.16,0.16}{#1}}
\newcommand{\AnnotationTok}[1]{\textcolor[rgb]{0.56,0.35,0.01}{\textbf{\textit{#1}}}}
\newcommand{\AttributeTok}[1]{\textcolor[rgb]{0.13,0.29,0.53}{#1}}
\newcommand{\BaseNTok}[1]{\textcolor[rgb]{0.00,0.00,0.81}{#1}}
\newcommand{\BuiltInTok}[1]{#1}
\newcommand{\CharTok}[1]{\textcolor[rgb]{0.31,0.60,0.02}{#1}}
\newcommand{\CommentTok}[1]{\textcolor[rgb]{0.56,0.35,0.01}{\textit{#1}}}
\newcommand{\CommentVarTok}[1]{\textcolor[rgb]{0.56,0.35,0.01}{\textbf{\textit{#1}}}}
\newcommand{\ConstantTok}[1]{\textcolor[rgb]{0.56,0.35,0.01}{#1}}
\newcommand{\ControlFlowTok}[1]{\textcolor[rgb]{0.13,0.29,0.53}{\textbf{#1}}}
\newcommand{\DataTypeTok}[1]{\textcolor[rgb]{0.13,0.29,0.53}{#1}}
\newcommand{\DecValTok}[1]{\textcolor[rgb]{0.00,0.00,0.81}{#1}}
\newcommand{\DocumentationTok}[1]{\textcolor[rgb]{0.56,0.35,0.01}{\textbf{\textit{#1}}}}
\newcommand{\ErrorTok}[1]{\textcolor[rgb]{0.64,0.00,0.00}{\textbf{#1}}}
\newcommand{\ExtensionTok}[1]{#1}
\newcommand{\FloatTok}[1]{\textcolor[rgb]{0.00,0.00,0.81}{#1}}
\newcommand{\FunctionTok}[1]{\textcolor[rgb]{0.13,0.29,0.53}{\textbf{#1}}}
\newcommand{\ImportTok}[1]{#1}
\newcommand{\InformationTok}[1]{\textcolor[rgb]{0.56,0.35,0.01}{\textbf{\textit{#1}}}}
\newcommand{\KeywordTok}[1]{\textcolor[rgb]{0.13,0.29,0.53}{\textbf{#1}}}
\newcommand{\NormalTok}[1]{#1}
\newcommand{\OperatorTok}[1]{\textcolor[rgb]{0.81,0.36,0.00}{\textbf{#1}}}
\newcommand{\OtherTok}[1]{\textcolor[rgb]{0.56,0.35,0.01}{#1}}
\newcommand{\PreprocessorTok}[1]{\textcolor[rgb]{0.56,0.35,0.01}{\textit{#1}}}
\newcommand{\RegionMarkerTok}[1]{#1}
\newcommand{\SpecialCharTok}[1]{\textcolor[rgb]{0.81,0.36,0.00}{\textbf{#1}}}
\newcommand{\SpecialStringTok}[1]{\textcolor[rgb]{0.31,0.60,0.02}{#1}}
\newcommand{\StringTok}[1]{\textcolor[rgb]{0.31,0.60,0.02}{#1}}
\newcommand{\VariableTok}[1]{\textcolor[rgb]{0.00,0.00,0.00}{#1}}
\newcommand{\VerbatimStringTok}[1]{\textcolor[rgb]{0.31,0.60,0.02}{#1}}
\newcommand{\WarningTok}[1]{\textcolor[rgb]{0.56,0.35,0.01}{\textbf{\textit{#1}}}}
\usepackage{graphicx}
\makeatletter
\def\maxwidth{\ifdim\Gin@nat@width>\linewidth\linewidth\else\Gin@nat@width\fi}
\def\maxheight{\ifdim\Gin@nat@height>\textheight\textheight\else\Gin@nat@height\fi}
\makeatother
% Scale images if necessary, so that they will not overflow the page
% margins by default, and it is still possible to overwrite the defaults
% using explicit options in \includegraphics[width, height, ...]{}
\setkeys{Gin}{width=\maxwidth,height=\maxheight,keepaspectratio}
% Set default figure placement to htbp
\makeatletter
\def\fps@figure{htbp}
\makeatother
\setlength{\emergencystretch}{3em} % prevent overfull lines
\providecommand{\tightlist}{%
  \setlength{\itemsep}{0pt}\setlength{\parskip}{0pt}}
\setcounter{secnumdepth}{-\maxdimen} % remove section numbering
\ifLuaTeX
  \usepackage{selnolig}  % disable illegal ligatures
\fi
\usepackage{bookmark}
\IfFileExists{xurl.sty}{\usepackage{xurl}}{} % add URL line breaks if available
\urlstyle{same}
\hypersetup{
  pdftitle={Cattle Emission Reduction},
  pdfauthor={Greys Otiniano},
  hidelinks,
  pdfcreator={LaTeX via pandoc}}

\title{Cattle Emission Reduction}
\author{Greys Otiniano}
\date{2025-06-17}

\begin{document}
\maketitle

\subsection{Scope}\label{scope}

The livestock sector is a cornerstone of rural livelihoods and national
food security in Kenya. However, prevailing production
systems---characterized by extended animal rearing cycles, inefficient
feeding, and informal marketing---contribute significantly to greenhouse
gas (GHG) emissions and environmental degradation. These inefficiencies
also undermine profitability for producers and create inconsistent
supply for meat processors and retailers.

As climate change intensifies, there is a growing need to transition to
more efficient, climate-smart livestock production systems. One
high-potential opportunity is to shorten livestock holding periods
through improved feeding, finishing, and offtake practices---reducing
methane emissions while increasing returns for producers and processors.

A meat processor aims to pioneer a contracted green beef production
model, working directly with livestock producers and traders to
encourage the supply of younger, better-finished animals. This model
aligns with global shifts toward sustainable sourcing, low-carbon value
chains, and inclusive business models.

\begin{Shaded}
\begin{Highlighting}[]
\NormalTok{input\_estimates }\OtherTok{\textless{}{-}} \FunctionTok{data.frame}\NormalTok{(}\AttributeTok{variable =} \FunctionTok{c}\NormalTok{(}\StringTok{"Infrastructure\_cost"}\NormalTok{, }\StringTok{"Establishment\_labor\_cost"}\NormalTok{, }\StringTok{"Labor\_cost"}\NormalTok{, }
                                           \StringTok{"Management\_cost"}\NormalTok{, }\StringTok{"Biogas\_profit"}\NormalTok{,}
                                           \StringTok{"Cattle\_profit"}\NormalTok{),}
                              \AttributeTok{lower =} \FunctionTok{c}\NormalTok{(}\DecValTok{5000}\NormalTok{, }\DecValTok{14500}\NormalTok{, }\DecValTok{5000}\NormalTok{, }\DecValTok{4000}\NormalTok{, }\DecValTok{3500}\NormalTok{, }\DecValTok{1500}\NormalTok{),}
                              \AttributeTok{median =} \ConstantTok{NA}\NormalTok{,}
                              \AttributeTok{upper =} \FunctionTok{c}\NormalTok{(}\DecValTok{15000}\NormalTok{, }\DecValTok{30000}\NormalTok{, }\DecValTok{15000}\NormalTok{,}\DecValTok{5000}\NormalTok{, }\DecValTok{6500}\NormalTok{, }\DecValTok{25000}\NormalTok{),}
                              \AttributeTok{distribution =} \FunctionTok{c}\NormalTok{(}\StringTok{"posnorm"}\NormalTok{, }\StringTok{"posnorm"}\NormalTok{, }\StringTok{"posnorm"}\NormalTok{, }
                                               \StringTok{"posnorm"}\NormalTok{, }\StringTok{"posnorm"}\NormalTok{, }\StringTok{"posnorm"}\NormalTok{),}
                              \AttributeTok{label =} \FunctionTok{c}\NormalTok{(}\StringTok{"Infrastructure (USD)"}\NormalTok{, }\StringTok{"Establishment labor cost (USD/ha)"}\NormalTok{, }\StringTok{"Labor cost(USD/ha)"}\NormalTok{,}
                                        \StringTok{"Management cost (USD/ha)"}\NormalTok{, }\StringTok{"Biogas cost (USD/m3)"}\NormalTok{,}
                                        \StringTok{"Cattle cost (USD/ha)"}\NormalTok{),}
                              \AttributeTok{Description =} \FunctionTok{c}\NormalTok{(}\StringTok{"Infrastructure cost"}\NormalTok{,}
                                              \StringTok{"Establishment labor cost"}\NormalTok{,}
                                              \StringTok{"Labor costs in a normal season"}\NormalTok{, }
                                              \StringTok{"Management costs in a normal season"}\NormalTok{,}
                                              \StringTok{"Biogas profits"}\NormalTok{,}
                                              \StringTok{"Cattle profits"}\NormalTok{))}

\NormalTok{model\_function }\OtherTok{\textless{}{-}} \ControlFlowTok{function}\NormalTok{()\{}
  
  \CommentTok{\# Estimate the income in a normal season}
\NormalTok{  income }\OtherTok{\textless{}{-}}\NormalTok{ Biogas\_profit }\SpecialCharTok{+}\NormalTok{ Cattle\_profit}
  
\NormalTok{  overall\_costs }\OtherTok{\textless{}{-}}\NormalTok{ Infrastructure\_cost }\SpecialCharTok{+}\NormalTok{ Establishment\_labor\_cost }\SpecialCharTok{+}\NormalTok{ Labor\_cost }\SpecialCharTok{+}
\NormalTok{                                           Management\_cost}
  
  \CommentTok{\# Estimate the final results from the model}
\NormalTok{  final\_result }\OtherTok{\textless{}{-}}\NormalTok{ income }\SpecialCharTok{{-}}\NormalTok{ overall\_costs}
  
  \CommentTok{\# Generate the list of outputs from the Monte Carlo simulation}
  \FunctionTok{return}\NormalTok{(}\FunctionTok{list}\NormalTok{(}\AttributeTok{final\_result =}\NormalTok{ final\_result))}
\NormalTok{\}}

\DocumentationTok{\#\# ?mcsimulation {-} to look what it is for\}}
\DocumentationTok{\#\# install.packages("decisionSupport")}
\FunctionTok{library}\NormalTok{(decisionSupport)}
\DocumentationTok{\#\# Run the Monte Carlo simulation using the model function}
\NormalTok{example\_mc\_simulation }\OtherTok{\textless{}{-}} \FunctionTok{mcSimulation}\NormalTok{(}\AttributeTok{estimate =} \FunctionTok{as.estimate}\NormalTok{(input\_estimates),}
                                      \AttributeTok{model\_function =}\NormalTok{ model\_function,}
                                      \AttributeTok{numberOfModelRuns =} \DecValTok{30000}\NormalTok{,}
                                    \AttributeTok{functionSyntax =} \StringTok{"plainNames"}\NormalTok{)}

  \DocumentationTok{\#\# functionSyntax {-} we can use plainnames of the variables we are using; speeding up the simulation process}

  \DocumentationTok{\#\#example\_mc\_simulation}
                                

\DocumentationTok{\#\# Graphs}
\end{Highlighting}
\end{Shaded}

\begin{Shaded}
\begin{Highlighting}[]
\DocumentationTok{\#\# designed for decisionsupport package, is not gonna run without the prior}
\DocumentationTok{\#\# by vars "final result" is meant}
\FunctionTok{plot\_distributions}\NormalTok{(}\AttributeTok{mcSimulation\_object =}\NormalTok{ example\_mc\_simulation,}
                   \AttributeTok{vars =} \StringTok{"final\_result"}\NormalTok{,}
                   \AttributeTok{method =} \StringTok{"boxplot\_density"}\NormalTok{,}
                   \AttributeTok{old\_names =} \StringTok{"final\_result"}\NormalTok{,}
                   \AttributeTok{new\_names =} \StringTok{"Outcome distribution for profits"}\NormalTok{)}
\end{Highlighting}
\end{Shaded}

\includegraphics{S6_Homework_Cattle_files/figure-latex/unnamed-chunk-2-1.pdf}

\end{document}
